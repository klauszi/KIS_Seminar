\section{Schluss}

\subsection{Zusammenfassung und Ausblick}

\begin{frame}{\insertsubsection}
  \begin{block}<+->{Zusammenfassung}
    \begin{itemize}
      \item Inkrementelle Reparatur der Daten durch Anomalienerkennungsverfahren 
      \item Minimum-Change-Prinzip und tempor�re Eigenschaften 
      \item Bessere Laufzeit durch Matrix-Pruning 
      \item IC Parameter Sch�tzung in $O(1)$
      \item H�here Genauigkeit der Reparatur als State-of-the-art Verfahren
    \end{itemize}
  \end{block}
  \begin{block}<+->{Ausblick}
    \begin{itemize}
      \item Andere Modelle anstatt ARX  
    \end{itemize}
  \end{block}
\end{frame}
\begin{frame}
\begin{center}
\huge{Vielen Dank\\ f�r die Aufmerksamkeit}
\end{center}
\end{frame}




\subsection{Literatur}

\begin{frame}[allowframebreaks]{\insertsubsection}
  \begingroup
  \small
  \beamertemplatebookbibitems
  \bibliographystyle{plain}
  \bibliography{Beispiel}
  \endgroup
\end{frame}


